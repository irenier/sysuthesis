% !TeX root = ../thesis.tex

\chapter{测试}

\section{一级节标题}

\subsection{二级节标题}

\subsubsection{三级节标题}

劳仑衣普桑,认至将指点效则机,最你更枝。想极整月正进好志次回总般,段然取向使张规军证回,世市总李率英茄持伴。用阶千样响领交出,器程办管据家元写,名其直金团。化达书据始价算每百青,金低给天济办作照明,取路豆学丽适市确。如提单各样备再成农各政,设头律走克美技说没,体交才路此在杠。响育油命转处他住有,一须通给对非交矿今该,花象更面据压来。与花断第然调,很处己队音,程承明邮。常系单要外史按机速引也书,个此少管品务美直管战,子大标蠢主盯写族般本。农现离门亲事以响规,局观先示从开示,动和导便命复机李,办队呆等需杯。见何细线名必子适取米制近,内信时型系节新候节好当我,队农否志杏空适花。又我具料划每地,对算由那基高放,育天孝。派则指细流金义月无采列,走压看计和眼提问接,作半极水红素支花。果都济素各半走,意红接器长标,等杏近乱共。层题提万任号,信来查段格,农张雨。省着素科程建持色被什,所界走置派农难取眼,并细杆至志本。

\section{脚注}

Lorem ipsum dolor sit amet, consectetuer adipiscing elit. Ut purus elit,
vestibulum ut, placerat ac, adipiscing vitae, felis. Curabitur dictum
gravida mauris. Nam arcu libero, nonummy eget, consectetuer id,
vulputate a, magna. Donec vehicula augue eu neque. Pellentesque habitant
morbi tristique senectus et netus et malesuada fames ac turpis egestas.
Mauris ut leo. Cras viverra metus rhoncus sem. Nulla et lectus
vestibulum urna fringilla ultrices. Phasellus eu tellus sit amet tortor
gravida placerat. Integer sapien est, iaculis in, pretium quis, viverra
ac, nunc. Praesent eget sem vel leo ultrices bibendum. Aenean faucibus.
Morbi dolor nulla, malesuada eu, pulvinar at, mollis ac, nulla.
Curabitur auctor semper nulla. Donec varius orci eget risus. Duis nibh
mi, congue eu, accumsan eleifend, sagittis quis, diam. Duis eget orci
sit amet orci dignissim rutrum.\footnote{这是一个脚注。}

\section{浮动体}

\subsection{三线表}

三线表使用如下格式,如\autoref{tab-tab}。

\begin{table}[H]
	\centering
	\caption{表号和表题在表的正上方}
	\label{tab-tab}
	\begin{tabular}{ll}
		\toprule
		说明     & 说明     \\
		\midrule
		这是一个测试 & 这是一个测试 \\
		这是一个测试 & 这是一个测试 \\
		\bottomrule
	\end{tabular}
\end{table}

\subsection{插图}

插入图片于固定位置,如\autoref{fig-single}。
\begin{figure}[H]
	\centering
	\includegraphics[width=3cm]{figures/sysu-badge.pdf}
	\caption{插图}
	\label{fig-single}
\end{figure}

对于图片的并排,推荐使用 \verb|subcaption| 宏包。

\begin{figure}[H]
	\centering
	\begin{subfigure}{3cm}
		\centering
		\includegraphics[width=3cm]{figures/sysu-badge.pdf}
		\caption{校徽}
	\end{subfigure}
	\hspace{3cm}
	\begin{subfigure}{3cm}
		\centering
		\includegraphics[width=3cm]{figures/sysu-badge.pdf}
		\caption{校徽}
	\end{subfigure}
	\caption{插图}
	\label{fig-example}
\end{figure}

\subsection{算法}

插入算法建议使用 \verb|algorithm2e| 宏包。
\vspace{.5\baselineskip}

\begin{algorithm}[H]
	\KwIn{input data}
	\KwOut{output data}
	\tcc{a comment line in C-style}
	\Repeat{$e<\tau$}{
		$f_n\leftarrow Y_1$\;
		$f_{n+1}\leftarrow f_n\times f_{n-1}$\;
		$e\leftarrow \frac{f_n}{2}$\;
	}
	\KwRet{$e$}
	\caption{算法示例}
	\label{algo:algorithm1}
\end{algorithm}

\section{数学相关}

\subsection{数学符号和公式}

本模板使用 \verb|unicode-math| 宏包,并设置了 \verb|math-style=ISO|。需要注意以下几点:

\begin{enumerate}
	\item 大写希腊字母默认为斜体,如
	      \begin{equation*}
		      \Gamma \Delta \Theta \Lambda \Xi \Pi \Sigma \Upsilon \Phi \Psi \Omega.
	      \end{equation*}
	      若需使用正体,请在命令前加 \verb|up|,例如
	      \begin{verbatim}
              \upGamma \upDelta \upTheta \upLambda \upXi \upPi
              \upSigma \upUpsilon \upPhi \upPsi \upOmega
          \end{verbatim}
	      则可得到
	      \begin{equation*}
		      \upGamma \upDelta \upTheta \upLambda \upXi \upPi \upSigma \upUpsilon \upPhi \upPsi \upOmega.
	      \end{equation*}
	\item 偏微分符号 $\partial$ 默认使用斜体。若需默认使用正体,请在 \texttt{sysusetup.tex} 文件中的 \verb|\unimathsetup| 中设置 \verb|partial=upright|。
	\item 哈密顿算子 $\nabla$ 默认使用正体。
	      若需默认使用斜体,请在 \texttt{sysusetup.tex} 文件中的 \verb|\unimathsetup| 中设置 \verb|nabla=italic|。
\end{enumerate}

公式按章编独立序号:
\begin{equation} \label{eq-1}
	\int_I \omega(x)\diff x = \upint_{a}^b f(x)\diff x - \lowint_a^b f(x)\diff x.
\end{equation}
引用公式:见\autoref{eq-1}。

区分 \verb|\mathscr| 和 \verb|\mathcal|:
\begin{gather*}
	\mathscr{A} \mathscr{B} \mathscr{C} \mathscr{D} \mathscr{E} \mathscr{F} \mathscr{G} \mathscr{H} \mathscr{I} \mathscr{J} \mathscr{K} \mathscr{L} \mathscr{M} \mathscr{N} \mathscr{O} \mathscr{P} \mathscr{Q} \mathscr{R} \mathscr{S} \mathscr{T} \mathscr{U} \mathscr{V} \mathscr{W} \mathscr{X} \mathscr{Y} \mathscr{Z}\\
	\mathcal{A} \mathcal{B} \mathcal{C} \mathcal{D} \mathcal{E} \mathcal{F} \mathcal{G} \mathcal{H} \mathcal{I} \mathcal{J} \mathcal{K} \mathcal{L} \mathcal{M} \mathcal{N} \mathcal{O} \mathcal{P} \mathcal{Q} \mathcal{R} \mathcal{S} \mathcal{T} \mathcal{U} \mathcal{V} \mathcal{W} \mathcal{X} \mathcal{Y} \mathcal{Z}.
\end{gather*}

\subsection{定理环境}

示例文件中使用 \verb|amsthm| 宏包配置了定理、引理和证明等环境。环境名有\\ \textsf{theorem}, \textsf{assertion}, \textsf{axiom}, \textsf{corollary}, \textsf{lemma}, \textsf{proposition}, \textsf{assumption}, \textsf{definition}, \textsf{example}, \textsf{remark}。

\begin{assertion} \label{assertion-1}
	测试。
\end{assertion}

\begin{assumption} \label{assumption-1}
	测试。
\end{assumption}

\begin{axiom} \label{axiom-1}
	测试。
\end{axiom}

\begin{example} \label{example-1}
	测试。
\end{example}

\begin{proof}
	测试。
\end{proof}

\begin{proposition} \label{proposition-1}
	测试。
\end{proposition}

\begin{remark}
	测试。
\end{remark}

\begin{definition} \label{def-1}
	设\(f(x)\)是\(E\subset\mathbb{R}^n\)上的非负可测函数,定义\(f(x)\)在\(E\)上的积分为
	\[\int_E f(x)\diff x = \sup_{\stackrel{h(x)\leqslant f(x)}{x\in E}}\left\{\int_E h(x)\diff x : h(x)\,\text{是\(\mathbb{R}^n\)上的非负可测简单函数}\right\}.\]
	这里的积分可以是\(+\infty\);若\(\int_E f(x)\diff x < +\infty\),则称\(f(x)\)在\(E\)上是\textsf{可积的},或称\(f(x)\)是\(E\)上的\textsf{可积函数}。
\end{definition}

\begin{lemma}[Chebyshev 不等式] \label{lemma-1}
	若\(f(x)\)是\(E\)上的非负可测函数,且\(a\)为正的常数,则
	\[m\left(\left\{x\in E : f(x) \geqslant a\right\}\right)\leqslant \frac{1}{a}\int_{E} f(x)\diff x.\]
\end{lemma}

\begin{proof}
	\begin{align*}
		\int_E f(x)\diff x & \geqslant \int_E f(x)\chi_{E\left(f \geqslant a\right)}(x)\diff x = \int_{E\left(f \geqslant a\right)} f(x)\diff x             \\
		                   & \geqslant \int_{E\left(f\geqslant a\right)} a\diff x = a\cdot m\left(\left\{x\in E : f(x) \geqslant a\right\}\right). \qedhere
	\end{align*}
\end{proof}

\begin{theorem}[控制收敛定理] \label{thm-1}
	设\(f_k\in L(E)\) \((k = 1,2,\cdots)\),且有
	\[\lim_{k\to\infty}f_k(x) = f(x), \,\text{a.e.}\, x\in E.\]
	若存在\(E\)上的\textsf{可积}函数\(F(x)\),使得
	\[\left\vert f_k(x)\right\vert\leqslant F(x),\,\text{a.e.}\, x\in E\quad (k = 1,2,\cdots)\]
	则
	\[\lim_{k\to\infty} \int_E f_k(x)\diff x = \int_E f(x)\diff x.\]
\end{theorem}

\begin{corollary}[依测度收敛型控制收敛定理] \label{corollary-1}
	设\(f_k\in L(\mathbb{R}^n)\) \((k = 1,2,\cdots)\),且\(f_k(x)\)在\(\mathbb{R}^n\)上依测度收敛于\(f(x)\)。若存在\(F\in L(\mathbb{R}^n)\)使得
	\[\left\vert f_k(x)\right\vert \leqslant F(x),\,\text{a.e.}\, x\in\mathbb{R}^n \quad (k = 1,2,\cdots)\]
	则\(f\in L(\mathbb{R}^n)\),且有
	\[\lim_{k\to\infty}\int_{\mathbb{R}^n} f_k(x)\diff x = \int_{\mathbb{R}^n} f(x)\diff x.\]
\end{corollary}

引用定理编号:\autoref{thm-1},\autoref{assertion-1},\autoref{axiom-1},\autoref{corollary-1},\autoref{lemma-1},\autoref{proposition-1},\autoref{assumption-1}, \autoref{def-1},\autoref{example-1}。

引用定理名称:\nameref{lemma-1},\nameref{thm-1},\nameref{corollary-1}。

\section{引用文献的标注}

BIB\TeX 数据库文件位于 \verb|bib/sysu.bib|。

参考文献的序号左顶格,并用数字加方括号表示,如“[1]”。每一条参考文献著录均以“.”结束。

测试标注格式,如\autoref{tab-ref} 所示。其他的标注格式请参考 \href{http://mirrors.ctan.org/biblio/bibtex/contrib/gbt7714/gbt7714.pdf}{gbt7714} 或者
\href{http://mirrors.ctan.org/macros/latex/contrib/biblatex-contrib/biblatex-gb7714-2015/biblatex-gb7714-2015.pdf}{biblatex-gb7714-2015}。

\begin{table}[H]
	\centering
	\caption{标注格式}
	\label{tab-ref}
	\begin{tabular}{ll}
		\toprule
		命令                                            & 结果                                         \\
		\midrule
		% \verb|\citep{Knuth1984}|                      & 数字标记法\citep{Knuth1984}                     \\
		\verb|\cite{Knuth1984}|                       & 角注标记法\cite{Knuth1984}                      \\
		% \verb|\parencite{Knuth1984}| & 数字标记法\parencite{Knuth1984} \\
		\verb|\cite{Knuth1986,Knuth1984}|             & 引用两个\cite{Knuth1986,Knuth1984}             \\
		\verb|\cite{Knuth1984,Knuth1986,Lamport1994}| & 引用多个\cite{Knuth1984,Knuth1986,Lamport1994} \\
		\bottomrule
	\end{tabular}
\end{table}
