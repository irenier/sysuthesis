\documentclass{sysuthesis}

\sysusetup[option]
  {
    type = bachelor
    % type = master
    % type = doctor
  }
\sysusetup[style]
  {
    % cjk-font  = windows,
    % math-font = newcm
  }
\sysusetup[info]
  {
    title        = {标题第一行\\标题第二行},
    title*       = {first line\\ second line},
    keywords     = {中山大学, 论文模版, 本科生, 研究生},
    keywords*    = {SYSU, template, undergraduate, graduate},
    student-id   = {00000000},
    author       = {作者名},
    author*      = {XXX},
    department   = {数学学院},
    department*  = {Department of Mathematics},
    major        = {数学与应用数学},
    major*       = {Mathematics and Applied Mathematics},
    supervisors  = {XXX~教授, XXX~教授},
    supervisors* = {Prof.~XXX, Prof.~XXX},
    date         = {2025/05/31},  % 默认为编译当天
    secret-level = {公开},  % 仅研究生
    thesis-code  = {00000000}  % 仅研究生
  }

% 参考文献设置: 可选 BibTeX 或 BibLaTeX
\sysusetup[bib]
  {
    backend  = bibtex,
    style    = gbt7714-numerical,
    resource = sysuthesis-sample
  }
% \sysusetup[bib]
%   {
%     backend  = biblatex,
%     style    = gb7714-2015,
%     resource = sysuthesis-sample.bib
%   }

% 载入其他常用宏包
\usepackage{thmtools}       % 数学定理
\usepackage{subcaption}     % 图片并排
\usepackage{tabularray}     % 表格
\UseTblrLibrary{booktabs}   % 三线表样式
\usepackage{float}          % 浮动体
\usepackage[ruled]{algorithm2e} % 算法
\usepackage{minted}         % 代码

\begin{document}

\maketitle
\frontmatter

\begin{abstract}
  这里是中文摘要内容 \cite{Knuth1984}。
\end{abstract}

\begin{abstract*}
  Here is the English abstract content \cite{Knuth1986}。
\end{abstract*}

\tableofcontents

\mainmatter

\chapter{引言}

本章主要介绍研究背景、意义以及论文的整体结构。

\section{研究背景}

此处陈述研究背景和现状。脚注是个好用的功能。\footnote{这是一个脚注示例。}

\section{主要工作}

本文的主要工作包括……。脚注可以多次使用。\footnote{这是另一个脚注示例。}

\chapter{理论基础}

本章主要介绍研究所需的理论基础,包括相关定义、定理和关键数学公式。

\section{基本定义与定理}

下面将介绍本文所依赖的一些基本定理和定义。

\begin{lemma}[Fubini] \label{lemma-Fubini}
  这里是 Fubini 引理的具体内容。
\end{lemma}

\begin{theorem}[Lebesgue] \label{thm-Lebesgue}
  这里是 Lebesgue 定理的具体内容。
\end{theorem}

\begin{corollary}[Fourier] \label{cor-Fourier}
  这里是 Fourier 推论的具体内容。
\end{corollary}

\begin{proposition}[Cantor] \label{prop-Cantor}
  这里是 Cantor 命题的具体内容。
\end{proposition}

\begin{definition}[Riemann] \label{def-Riemann}
  这里是 Riemann 定义的具体内容。
\end{definition}

\begin{example}[Gauss] \label{exp-Gauss}
  这里是 Gauss 示例的具体内容。
\end{example}

\begin{remark}[Kolmogorov] \label{rmk-Kolmogorov}
  这里是 Kolmogorov 评注的具体内容。
\end{remark}

\begin{proof}
  这里是证明过程。
\end{proof}

通过 \verb|\autoref| 命令可以方便地引用这些环境,例如:\autoref{thm-Lebesgue}, \autoref{lemma-Fubini}, \autoref{cor-Fourier}, \autoref{prop-Cantor}, \autoref{def-Riemann}, \autoref{exp-Gauss}, \autoref{rmk-Kolmogorov}。

\section{关键数学公式}

大写希腊字母默认为斜体,例如:
\begin{equation*}
  \Gamma \Delta \Theta \Lambda \Xi \Pi \Sigma \Upsilon \Phi \Psi \Omega.
\end{equation*}

公式按章节独立编号,便于引用。
\begin{equation} \label{eq-1}
  \int_I \omega(x)\ dx = \upint_{a}^b f(x)\ dx - \lowint_a^b f(x)\ dx.
\end{equation}
引用上述公式:见\autoref{eq-1}。

\chapter{核心算法与实现}

本章介绍本文提出的核心算法及其代码实现。

\section{算法描述}

算法的具体流程如 \autoref{alg-algorithm} 所示。

\begin{algorithm}[H]
  \caption{算法流程示例}
  \label{alg-algorithm}
  \KwIn{输入数据}
  \KwOut{输出结果}
  \tcc{这是一个C语言风格的注释}
  \Repeat{$e < \tau$}{
    $f_n\leftarrow Y_1$\;
    $f_{n+1}\leftarrow f_n\times f_{n-1}$\;
    $e\leftarrow \frac{f_n}{2}$\;
  }
  \KwRet{$e$}
\end{algorithm}

\section{代码实现示例}

下面是使用 C 语言实现的 "hello, world" 程序片段。

\begin{minted}[bgcolor=gray!5!white, autogobble]{c}
  int main() {
    printf("hello, world");
    return 0;
  }
\end{minted}

\chapter{实验结果与分析}

本章通过图表展示实验结果,并进行分析。

\section{插图示例}

图 \ref{fig-1} 展示了并排插图的示例。

\begin{figure}[H]
  \centering
  \begin{subfigure}{0.45\linewidth}
    \includegraphics[width=\linewidth]{example-image-a}
    \caption{子图 a}
  \end{subfigure}
  \hfil
  \begin{subfigure}{0.45\linewidth}
    \includegraphics[width=\linewidth]{example-image-b}
    \caption{子图 b}
  \end{subfigure}
  \caption{并排图片示例}\label{fig-1}
\end{figure}

\section{表格示例}

表格是展示结构化数据的有效方式。如\autoref{tab-1} 展示了一个标准的三线表。

\begin{table}[H]
  \caption{标准三线表}\label{tab-1}
  \centering
  \begin{tblr}{llll}
    \toprule
    Alpha   & Beta  & Gamma   & Delta \\
    \midrule
    Epsilon & Zeta  & Eta     & Theta \\
    Iota    & Kappa & Lambda  & Mu    \\
    Nu      & Xi    & Omicron & Pi    \\
    \bottomrule
  \end{tblr}
\end{table}

此外,我们还可以创建带有颜色的表格,如\autoref{tab-2} 所示。

\begin{table}[H]
  \caption{彩色条纹表格}\label{tab-2}
  \centering
  \begin{tblr}{
      row{odd} = {bg=azure8},
      row{1} = {bg=azure3, fg=white, font=\sffamily},
    }
    Alpha         & Beta         & Gamma           \\
    Delta         & Epsilon      & Zeta            \\
    Eta           & Theta        & Iota            \\
    Kappa         & Lambda       & Mu              \\
    Nu Xi Omicron & Pi Rho Sigma & Tau Upsilon Phi \\
  \end{tblr}
\end{table}

\appendix
\chapter{附录A}

这里是附录内容。可以放置一些补充材料。

\begin{figure}[H]
  \centering
  \includegraphics[width=\linewidth]{example-image-a}
  \caption{附录中的图片}
\end{figure}

\begin{acknowledgements}
  这里是致谢内容。
\end{acknowledgements}

\end{document}