\documentclass{sysuthesis}

\sysusetup[option]
  {
    type = bachelor
    % type = master
    % type = doctor
  }
\sysusetup[style]
  {
    % cjk-font  = windows,
    math-font = newcm
  }
\sysusetup[info]
  {
    title        = {标题第一行\\标题第二行},
    title*       = {first line\\ second line},
    keywords     = {中山大学, 论文模版, 本科生, 研究生},
    keywords*    = {SYSU, template, undergraduate, graduate},
    student-id   = {00000000},
    author       = {作者名},
    author*      = {XXX},
    department   = {数学学院},
    department*  = {Department of Mathematics},
    major        = {数学与应用数学},
    major*       = {Mathematics and Applied Mathematics},
    supervisors  = {XXX~教授, XXX~教授},
    supervisors* = {Professor~XXX, Professor~XXX},
    % date         = {2025/04/30},  % 默认为今日
    secret-level = {公开},  % 研究生模版参数
    thesis-code  = {00000000}  % 研究生模版参数
  }
\sysusetup[bib]
  {
    backend  = bibtex,
    style    = gbt7714-numerical,
    resource = sysuthesis-sample
  }
% \sysusetup[bib]
%   {
%     backend  = biblatex,
%     style    = gb7714-2015,
%     resource = sysuthesis-sample.bib
%   }

% 载入其他宏包
% 数学定理
\usepackage{thmtools}
% \usepackage{keytheorems}
% 图片并排
\usepackage{subcaption}
% 算法
\usepackage[ruled]{algorithm2e}
% 代码
\usepackage{minted}

\begin{document}

\maketitle
\frontmatter

\begin{abstract}
  中文摘要 \cite{Knuth1984}
\end{abstract}

\begin{abstract*}
  abstract \cite{Knuth1986}
\end{abstract*}

\tableofcontents

\mainmatter

\chapter{测试}
\section{测试}
\subsection{定理测试}

测试。\footnote{测试。}

\begin{lemma}[Fubini] \label{lemma-Fubini}
  测试。
\end{lemma}

\begin{theorem}[Lebesgue] \label{thm-Lebesgue}
  测试。
\end{theorem}

\begin{corollary}[Fourier] \label{cor-Fourier}
  测试。
\end{corollary}

\begin{proposition}[Cantor] \label{prop-Cantor}
  测试。
\end{proposition}

\begin{definition}[Riemann] \label{def-Riemann}
  测试。
\end{definition}

\begin{example}[Gauss] \label{exp-Gauss}
  测试。
\end{example}

\begin{remark}[Kolmogorov] \label{rmk-Kolmogorov}
  测试。
\end{remark}

\begin{proof}
  测试。
\end{proof}

\autoref{thm-Lebesgue}, \autoref{lemma-Fubini}, \autoref{cor-Fourier}, \autoref{prop-Cantor}, \autoref{def-Riemann}, \autoref{exp-Gauss}, \autoref{rmk-Kolmogorov} \footnote{测试。}

\nameref{thm-Lebesgue}, \nameref{lemma-Fubini}, \nameref{cor-Fourier}, \nameref{prop-Cantor}, \nameref{def-Riemann}, \nameref{exp-Gauss}, \nameref{rmk-Kolmogorov}

\newpage

\section{脚注测试}

测试。 \footnote{测试。}

\chapter{测试}
\section{算法测试}

\begin{algorithm}
  \caption{算法示例}
  \label{alg-algorithm}
  \KwIn{input data}
  \KwOut{output data}
  \tcc{a comment line in C-style}
  \Repeat{$e<\tau$}{
    $f_n\leftarrow Y_1$\;
    $f_{n+1}\leftarrow f_n\times f_{n-1}$\;
    $e\leftarrow \frac{f_n}{2}$\;
  }
  \KwRet{$e$}
\end{algorithm}

\autoref{alg-algorithm}

\section{代码测试}

\begin{minted}[bgcolor=gray!5!white, autogobble]{c}
  int main() {
    printf("hello, world");
    return 0;
  }
\end{minted}

\appendix

\chapter{测试}
\section{测试}

\end{document}