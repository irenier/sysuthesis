% !TeX root = ./thesis.tex

\sysusetup{
  title                   = {中山大学学位论文模板示例文档},
  title*                  = {An example of thesis template for Sun Yat-sen University},
  author                  = {作者名},
  author*                 = {author},
  student-id              = {00000000},
  department              = {数学学院},
  department*             = {Department of Mathematics},
  speciality              = {数学与应用数学},
  speciality*             = {Mathematics and Applied Mathematics},
  supervisor              = {XXX~教授, XXX~教授},
  supervisor*             = {Prof. XXX, Prof. XXX},
  keywords                = {中山大学, 学位论文, 学士},
  keywords*               = {Sun Yat-sen University (SYSU), Thesis, Bachelor},
  % date                    = {2023-03-01},   % 默认为今日
  cover-title             = twoline,   % 默认为 oneline
  cover-title-firstline   = {第一行},
  cover-title-secondline  = {第二行},
  cover-title-firstline*  = {firstline},
  cover-title-secondline* = {secondline},
  % color                   = black,   % 默认为 sysugreen
  % print                   = twoside,   % 默认为 oneside
  % number                  = chinese,   % 默认为 arabic
}

% 设置英文字体
\IfFontExistsTF{Times New Roman}{
  \setmainfont{Times New Roman}
}{
  \setmainfont{texgyretermes}[
    Extension      = .otf,
    UprightFont    = *-regular,
    BoldFont       = *-bold,
    ItalicFont     = *-italic,
    BoldItalicFont = *-bolditalic
  ]
}
\setsansfont{texgyreheros}[
  Extension      = .otf,
  UprightFont    = *-regular,
  BoldFont       = *-bold,
  ItalicFont     = *-italic,
  BoldItalicFont = *-bolditalic
]
\setmonofont{inconsolata}

% 设置数学字体及相关格式
\unimathsetup{
  math-style=ISO,
  % partial=upright,
  % nabla=italic,
}

\setmathfont{XITS Math}
\setmathfont{XITS Math}[range={cal,bfcal},StylisticSet=01]

% \setmathfont{NewComputerModernMath}
% \setmathfont{NewComputerModernMath}[range={scr,bfscr},StylisticSet=01]

% 加载额外的宏包

% 定理类环境宏包
\usepackage{aliascnt}
\usepackage{amsthm}

% 插图
\usepackage{graphicx}

% 三线表
\usepackage{booktabs}

% 图片并排
\usepackage{subcaption}

% 跨页表格
\usepackage{longtable}

% 算法
\usepackage[ruled,linesnumbered]{algorithm2e}

% SI 量和单位
\usepackage{siunitx}

% 参考文献使用 BibTeX + natbib 宏包
% 顺序编码制
\usepackage[sort&compress]{gbt7714}
\bibliographystyle{gbt7714-numerical}
% 著者-出版年制
% \bibliographystyle{gbt7714-author-year}

% 参考文献使用 BibLaTeX 宏包
% \usepackage[backend=biber,style=gb7714-2015]{biblatex}
% \usepackage[backend=biber,style=gb7714-2015ay]{biblatex}
% 声明 BibLaTeX 的数据库
% \addbibresource{bib/sysu.bib}

% 数学命令
\newcommand\diff{\mathop{}\!\symup{d}}
\newcommand\eup{{\symup{e}}}
\newcommand\iup{{\symup{i}}}

% hyperref 宏包在最后调用
\usepackage{hyperref}

% 配置图片的默认目录
\graphicspath{{figures/}{pictures/}}
